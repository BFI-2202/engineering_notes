\documentclass{article}
\usepackage[utf8]{inputenc}

\usepackage[T2A]{fontenc}
\usepackage[utf8]{inputenc}
\usepackage[russian]{babel}

\usepackage{minted}
\newtheorem{definition}{Определение}


\title{Программная инженерия}
\author{Лисид Лаконский}
\date{February 2023}

\begin{document}
\raggedright

\maketitle
\tableofcontents
\pagebreak

\section{Программная инженерия - 13.02.2023}

\begin{definition}
    CASE = Computer-aided software engineering
\end{definition}

\subsection{Введение в автоматизированные системы}

\begin{definition}
    \textbf{Система} — способ принятия решений, при котором каждое последующее действие осуществляется по заранее заданной инструкции, правилу, алгоритму, методу, а не изобретается заново
\end{definition}

\begin{definition}
    \textbf{Автоматизированная система} — система, в которой задействованы человек и средства вычислительной техники
\end{definition}

\begin{definition}
    \textbf{Автоматическая система} — система, в которой человек отсутствует
\end{definition}

\begin{definition}
    \textbf{Информационная система} — система, созданная для хранения и обработки информации
\end{definition}

\begin{definition}
    \textbf{Производственная система} — система, разработанная для эксплуатации на производстве
\end{definition}

\begin{definition}
    \textbf{Мониторинг} — сбор информации
\end{definition}

\begin{definition}
    \textbf{Контроль} — сбор информации и сопоставление фрагментов информации с эталонным образцом и между собой. Контроль включает в себя \textbf{мониторинг}
\end{definition}

\begin{definition}
    \textbf{Управление} — сбор информации, сопоставление фрагментов информации между собой или с контрольным образцом и выдача объекту управления управляющего сигнала или команды
\end{definition}

\textbf{Главной задачей} автоматизированных систем является осуществление мониторинга, контроля и управления информационными и производственными процессами организации или предприятия

\subsubsection{Федеральный закон «Об информации, информационных технологиях и о защите информации»}

Федеральный закон от 27.07.2006 номер 149-ФЗ «\textbf{Об информации, информационных технологиях и о защите информации}». Несколько определений из него:

\begin{definition}
    \textbf{Информация} - сведения (сообщения, данные) независимо от формы их представления
\end{definition}

\begin{definition}
    \textbf{Информационные технологии} - процессы, методы поиска, сбора, хранения, обработки, предоставления, распространения информации и способы осуществления таких процессов и методов
\end{definition}

\begin{definition}
    \textbf{Информационная система} - совокупность содержащейся в базах данных информации и обеспечивающих ее обработку информационных технологий и технических средств
\end{definition}

\begin{definition}
    \textbf{Информационно-телекоммуникационная сеть} - технологическая система, предназначенная для передачи по линиям связи информации, доступ к которой осуществляется с использованием средств вычислительной техники
\end{definition}

\begin{definition}
    \textbf{Информатизация} — поиск, получение, передача, производство и распространение информации; применение информационных технологий
\end{definition}

\begin{definition}
    \textbf{Электронное сообщение} — информация, переданная или полученная пользователем информационно-телекоммуникационной сети 
\end{definition}

\begin{definition}
    \textbf{Электронный документ} - документированная информация, представленная в электронной форме, то есть в виде, пригодном для восприятия человеком с использованием электронных вычислительных машин, а также для передачи по информационно-телекоммуникационным сетям или обработки в информационных
\end{definition}

\subsubsection{Другие федеральные (российской федерации) законы}

\begin{enumerate}
    \item Федеральный закон от 27.07.2006 номер 152-ФЗ «О персональных данных»
    \item Федеральный закон от 29.07.2004 номер 98-ФЗ «О коммерческой тайне»
    \item Закон РФ от 21.07.1993 номер 5485-1 «О государственной тайне»
    \item Федеральный закон от 06.04.2011 номер 63-ФЗ «Об электронной подписи»
    \item Федеральный закон от 27.12.2002 номер 184-ФЗ «О техническом регулировании»
    \item Федеральный закон от 04.05.2011 номер 99-ФЗ «О лицензировании отдельных видов деятельности»
\end{enumerate}

\subsubsection{Лирическое отступление: об электронной подписи}

\begin{definition}
    \textbf{Электронная подпись} - информация в электронной форме, которая присоединена к другой информации в электронной форме (подписываемой информации) или иным образом связана с такой информацией и которая используется для определения лица, подписывающего информацию
\end{definition}

\subsubsection{Федеральные службы}

\begin{enumerate}
    \item Федеральная служба безопасности (ФСБ)
    \item Федеральная служба по надзору в сфере связи, информационных технологий и массовых коммуникаций (Роскомнадзор)
\end{enumerate}

\end{document}