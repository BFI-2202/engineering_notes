\documentclass{article}
\usepackage[utf8]{inputenc}

\usepackage[T2A]{fontenc}
\usepackage[utf8]{inputenc}
\usepackage[russian]{babel}

\newtheorem{definition}{Определение}
\usepackage{multienum}
\usepackage{geometry}

\geometry{
	left=1cm,right=1cm,
	top=2cm,bottom=2cm
}

\title{Программная инженерия}
\author{Лисид Лаконский}
\date{February 2023}

\begin{document}
\raggedright

\maketitle
\tableofcontents
\pagebreak

\section{Программная инженерия - 22.02.2023}

\subsection{Введение в автоматизированные системы}

\textbf{Федеральное агентство по техническому регулированию и метрологии} (РосСтандарт) отличает за ввод на территории Российской Федерации государственных стандартов, межгосударственных стандартов и международных стандартов

\textbf{American National Standard Institute} (ANSI), \textbf{International Organization for Standardization}

\hfill

Стандарты призваны унифицировать производство и параметры продукции с целью более эффективной сборки готовых изделий из отдельных деталей, унифицировать контроль качества продукции, защищать потребителей от некачественной продукции

\hfill

Разработчику автоматизированных систем необходимо знать следующие группы ГОСТов:

\begin{enumerate}
	\item \textbf{ГОСТы 34-ой группы} определяют нормы разработки автоматизированных систем
	\item \textbf{ГОСТы 19-ой группы} определяют порядок проектирования и разработки программного обеспечения
	\item \textbf{ГОСТы 2-ой группы} определяют порядок разработки конструкторской документации, проектной документации, эксплуатационной документации
\end{enumerate}

\textbf{ГОСТ 34.003—90} содержит основные определения, связанные с автоматизированными системами:

\begin{enumerate}
	\item \textbf{Система} — совокупность элементов, объединенная связями между ними, и обладающая определенной целостностью
	\item \textbf{Информационные технологии} — приемы, способы и методы применения средств вычислительной техники при выполнении функций сбора, хранения, обработки, передачи и использования данных
	\item \textbf{Объект деятельности} — объект (процесс), состояние которого определяется поступающими на него воздействиями человека (коллектива) и, возможно, внешней среды
	\item \textbf{Управление} — совокупность целенаправленных действий, включающая оценку ситуации и состояния объекта управления, выбор управляющих воздействий и их реализация
	\item \textbf{Автоматизированная система} — это система, состоящая из персонала и комплекса средств автоматизации его деятельности, реализующая информационную технологию выполнения установленных функций. При этом, в зависимости от вида деятельности, выделяют, например, следующие виды АС:
	\begin{enumerate}
		\item \textbf{Автоматизированная система управления (АСУ)} — система человек-машина, обеспечивающая эффективное функционирование объекта, в которой сбор и переработка информации, необходимый для реализации функций управления, обеспечивается с использованием средств автоматизации и вычислительной техники

		Задачи:
		\begin{enumerate}
			\item Обеспечить работников органов управления достоверной социальной, экономической и другой информацией для повышения эффективности управления хозяйственными объектами
		\end{enumerate}
		\item \textbf{Система автоматизированного проектирования (САПР)}
		\item \textbf{Автоматизированная система научных исследований (АСНИ)}
	\end{enumerate}
\end{enumerate}

В зависимости от вида управляемого объекта (процесса) \textbf{АСУ} делят, например, на \textbf{АСУ технологическими процессами (АСУ ТП)}, \textbf{АСУ предприятиями (АСУ П)} и так далее. \\[1 mm]

\textbf{Современная большая АИС} — сложный комплекс технических и программных средств, технологии персонала, функции которого заключаются в сборе, обработке, хранении, поиске и выдаче пользователю информации о некоторой предметной области. \\[1 mm]

\textbf{Основные задачи АИС} состоят в реализации функций массового ввода данных в ЭВМ, их размещения в долговременной памяти, коррекции и удаления в случае необходимости, обеспечения доступа к данным для обслуживания запросов пользователей. \\[1 mm]

\begin{definition}

Под \textbf{автоматизированным банком данных} (БнД) понимается организационно-техническая система, представляющая собой совокупность баз данных пользователей, технических и программных средств формирования и ведения этих баз, и коллектива специалистов, обеспечивающих функционирование системы. \\[1mm]

Здесь \textbf{базу данных} (БД) можно определить как совокупность взаимосвязанных, хранящихся вместе данных, при наличии такой минимальной избыточности, которая допускает их использование оптимальным образом для одного или нескольких приложений. Данные запоминаются так, чтобы они были независимы от программы, использующей эти данные, и структурируются таким образом, чтобы была обеспечена возможность их дальнейшего наращивания.ire

\end{definition}

\begin{definition}

\textbf{Автоматизированный банк данных} — автоматизированная информационно-поисковая система, состоящая из одной или нескольких баз данных и системы хранения, обработки и поиска информации в них. \\[1mm]

Здесь \textbf{база данных} — набор данных, который достаточен для установленной цели, и представлен на машинном носителе в виде, позволяющем осуществлять автоматизированную обработку содержащейся в нем информации. \\[1mm]

А \textbf{информационно-поисковая система} — совокупность информационного фонда и технических средств информационного поиска в нем.

\end{definition}

\begin{definition}

\textbf{Информационная система персональных данных} — совокупность содержащихся в базе данных персональных данных и обеспечивающих их обработку информационных технологий и вычислительных средств. \\[1mm]

Здесь \textbf{персональные данные} — любая информация, относящая прямо или косвенно к определенному или определяемому физическому лицу (субъекту персональных данных)

\end{definition}

Автоматизированные системы также делят на \textbf{фотографические} и \textbf{документальные}:

\begin{enumerate}
	\item \textbf{Фотографические системы} предназначены для хранения и обработки структурированных данных в виде чисел и текстов
	\item В \textbf{документальных системах} информация представлена в виде документов, состоящих из наименований, описаний, ссылок и текстов. Поиск по неструктурированным данным с использованием семантических признаков, отобранные документы предоставляются пользователю, а обработка данных в таких системах практически не производится
\end{enumerate}

\subsubsection{Типовая структура автоматизированной системы}

\begin{minipage}[t]{0.45\textwidth}
\paragraph{Человек-машина, часть «машина»}

\begin{enumerate}
	\item Компьютерная техника, средства локальной вычислительной сетей и VPN-сетей: персональные компьютеры, серверы, маршрутизаторы, криптошлюзы и так далее
	\item На каждый компьютер устанавливается операционная система: Windows X, Windows X Server, Unix, Linux и другие
	\item На операционную систему устанавливаются сервера баз данных: Oracle, Microsoft SQL Server, PostgreSQL и другие; а также инструментальные системы: Oracle Developer Suite, Microsoft Visual Studio и другие.

	Сервера баз данных устанавливаются на физические сервера, а инструментальные системы — на компьютеры разработчиков
	\item Прикладное программное обеспечение, например 1С Бухгалтерия
\end{enumerate}
\end{minipage}%
\hfill
\begin{minipage}[t]{0.45\textwidth}
\paragraph{Человек-машина, часть «человек»}

\begin{enumerate}
	\item Снабженцы и материально-ответственные лица: системные администраторы, системные программисты
	\item Администраторы баз данных (backend programming)
	\item Администраторы баз данных, прикладные программисты
	\item Прикладные программисты (frontend programming), эксперты: например бухгалтера, врачи, юристы, банкиры, складские работники и так далее
\end{enumerate}
\end{minipage}

\hfill

\textbf{Компьютерная совместимость} подразумевает два аспекта: \textbf{аппаратную} и \textbf{программную}. Советская промышленность производила компьютеры \textbf{серии ЭС} и \textbf{серии ЭМ}; компания Apple производит \textbf{компьютеры Macintosh}; Digital Equipment Corporation производила \textbf{компьютеры VAX}

\hfill

Сетевая архитектура Ethernet \textbf{изначально имела три вида}:

\begin{multienumerate}
	\mitemxxx{На толстом коаксиальном кабеле}{На тонком коаксиальном кабеле}{На витой паре}
\end{multienumerate}

Сети делятся на \textbf{одноранговые} и \textbf{сети с выделенным файл-сервером}

\end{document}